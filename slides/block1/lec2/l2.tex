\input{../../common/slide-common-header.tex}

\newcommand{\orgNum}{0}
\newcommand{\orgTopic}{org meeting}
\newcommand{\orgKey}{syllabus, contacts}

\newcommand{\introNum}{1}
\newcommand{\introTopic}{introduction to multithreading}
\newcommand{\introKey}{concurrency, parallelism, agents, threads, scheduler, threading models, interleaving execution,  Amdahl's law, race condition, data race, deadlock, wait-for graph}

\newcommand{\basicNum}{2}
\newcommand{\basicTopic}{mutual exclusion}
\newcommand{\basicKey}{thread safety, concurrent consistency, mutual exclusion, critical section, deadlock-freedom, starvation, fairness, check-then-act, reentrancy, admission policy, code locking, data locking, lock splitting, lock ordering, dining philosophers problem}

\newcommand{\basicTwoNum}{3}
\newcommand{\basicTopic}{signalling}
\newcommand{\basicKey}{signalling, condition variable, lost signal, spurious wakeup}

\newcommand{\syncPrimitivesNum}{4}
\newcommand{\syncPrimitivesTopic}{advanced synchronization primitives}
\newcommand{\syncPrimitivesKey}{monitor, latch, barrier, thundering herd, semaphore, read-write lock, thread pool, executor, producer-consumer, fork-join, load balancing}

\newcommand{\patternsNum}{5}
\newcommand{\patternsTopic}{advanced synchronization concepts}
\newcommand{\patternsKey}{interruption, cancellation, partitioning, privatization, replication, thread-local, ownership}

\newcommand{\extraBasicsNum}{6}
\newcommand{\extraBasicsTopic}{additional topics of practical concurrency}
\newcommand{\extraBasicsKey}{documenting protocols and classes, checking concurrent invariants, stress testing, execution trace analysis, estimating required testing effort, static and dynamic checks, scheduling randomization, model checking}

\newcommand{\foundationsNum}{7}
\newcommand{\foundationsTopic}{theoretical foundations of concurrency}
\newcommand{\foundationsKey}{timeline, events, precedence, 2-thread mutual exclusion, deadlock freedom, starvation freedom, N-thread mutual exclusion, sequential objects and specifications, concurrent objects, linearizability}

\newcommand{\foundationsPlusNum}{8}
\newcommand{\foundationsPlusTopic}{progress guarantees, concurrent operations hierarchy, consensus number}
\newcommand{\foundationsPlusKey}{obstruction-free, lock-free, wait-free, safe register, regular register, atomic register, register snapshot, consensus number}

\newcommand{\atomicsNum}{9}
\newcommand{\atomicsTopic}{introduction to atomics}
\newcommand{\atomicsKey}{read-modify-write, get-and-add, compare-and-swap, spin lock, lock-free stack, ABA problem}

% TODO: taxonomy of queues, 

\newcommand{\cacheCoherencyNum}{10}
\newcommand{\cacheCoherencyTopic}{cache coherency}
\newcommand{\cacheCoherencyKey}{cache memory hierarchy, cache coherency protocol, store-buffer, load-buffer, invalidate-queue, memory barrier, hardware memory model, weak memory model, litmus tests}

\newcommand{\langMMNum}{11}
\newcommand{\langMMTopic}{language memory model}
\newcommand{\langMMKey}{compiler optimizations, compiler barriers, language memory model, strict consistency, threads cannot be implemented as a library, visibility, volatile}

\newcommand{\advancedConcurrencyNum}{12}
\newcommand{\advancedConcurrencyTopic}{advanced locking}
\newcommand{\advancedConcurrencyKey}{Anderson Queue Lock, CLH, MCS, check-then-act, spin-then-park}

% TODO: work distribution, work stealing, taxonomy of parallel problems, single LIFO cell optimization, RAT

\newcommand{\userSpaceThreadingNum}{13}
\newcommand{\userSpaceThreadingTopic}{user-space threading}
\newcommand{\userSpaceThreadingKey}{berkley socket, blocking and non-blocking IO, callback-hell, async-await, continuation-passing-style, fibers/coroutines/green threads, stackful vs stackless}

\newcommand{\concurrentQueuesNum}{14}
\newcommand{\concurrentQueuesNumTopic}{concurrent queues}
\newcommand{\concurrentQueuesNumKey}{queue as mutex, strong FIFO, per-producer FIFO, bounded/unbounded queue, SPSC, MPSC, SPMC, MPMC, total/partial/synchronous API, lock-free queue, ring buffer for bounded queues, concurrent dequeue, work stealing}

% \newcommand{\designNum}{13}
% \newcommand{\designTopic}{designing concurrent systems}
% \newcommand{\designKey}{park/unpark, synchronizer, futex/wait-on-address, plan9 approach, race-finders, ForkJoinPool/CoroutineCarriers/UIthread, observability, structured concurrency}

% \newcommand{\frameworksAndDistributedNum}{14}
% \newcommand{\frameworksAndDistributedTopic}{multi-agent systems}
% \newcommand{\frameworksAndDistributedKey}{auto-parallelization languages and frameworks, semi-automatic synchronization, distributed systems, consensus protocols}


\title[]{Lecture \basicNum: \basicTopic}
\subtitle[]{\basicKey}
\author[]{Alexander Filatov\\ filatovaur@gmail.com}

\date{}

\newcommand{\taskAsyncException}{2.1}
\newcommand{\taskProofMutex}{2.2.a}
\newcommand{\taskProofDeadlockFreedom}{2.2.b}
\newcommand{\taskReentrant}{2.3}
\newcommand{\taskCounters}{2.4}
\newcommand{\taskEmpireLock}{2.5}
\newcommand{\taskCodeDining}{2.6}

\begin{document}

\begin{frame}
  \titlepage
    \url{https://github.com/Svazars/parallel-programming/blob/main/slides/pdf/l2.pdf}
\end{frame}

\begin{frame}{In previous episode}

\begin{itemize}
 \item We study communication and coordination of different agents.
 \item Every agent has its own speed and scenario of execution.
 \item We are focusing on threads which are part of OS process and managed by scheduler.
 \item We expect OS to use pre-emptive multitasking (time-sharing of CPU cores).
 \item Interleaving \texttt{N:1} is useful yet simplified model of concurrent execution.
\end{itemize}

Any concurrent task has
\begin{itemize}
    \item parallel (independent) and sequential (dependent)
\end{itemize}
parts so max speedup is limited by Amdahl's law.

Threads have read/write access to shared memory which leads to
\begin{itemize}
    \item non-determinism, race conditions, data races, visibility problems
\end{itemize}

Threads use blocking methods which leads to
\begin{itemize}
    \item deadlocks, priority inversion
\end{itemize}
therefore we use wait-for graphs and observability API.

\end{frame}

\begin{frame}{Lecture plan}
\tableofcontents
\end{frame}

\section{Thread safety}
\showTOC

\begin{frame}[t,fragile]{Toy problem: thread-safe counter}
\framesubtitle{Description}

\begin{minted}{java}
public class Counter {
    public Counter(long initial) { ... }
    public void increment() { ... }
    public long get() { ... }
}
\end{minted}
\end{frame}

\questiontime{What is ''thread-safe''?}

\begin{frame}[noframenumbering,t,fragile]{Toy problem: thread-safe counter}
\framesubtitle{Description}

\begin{minted}{java}
public class Counter {
    public Counter(long initial) { ... }
    public void increment() { ... }
    public long get() { ... }
}
\end{minted}

Thread-safe -- may be invoked from different threads simultaneously and behave ''normally''.

\pause
What is normal?

\pause
\begin{itemize}
    \item \texttt{get} and \texttt{increment}} are consistent 
    \item no \texttt{increment} is lost
\end{itemize}
\end{frame}


\begin{frame}[t,fragile]{Toy problem: thread-safe counter}
\framesubtitle{Description}

\begin{minted}{java}
public class Counter {
    public Counter(long initial) { ... }
    public void increment() { ... }
    public long get() { ... }
}
\end{minted}

How to handle race conditions?

\pause

How to distinguish user-side misuse from library-side bug?

\end{frame}

\begin{frame}[t,fragile]{Toy problem: thread-safe counter}
\framesubtitle{Description}

\begin{minted}{java}
Counter c = new Counter(0);
Thread t1 = new Thread( () -> { c.increment(); println(c.get()); });
Thread t2 = new Thread( () -> { c.increment(); println(c.get()); });
t1.start(); t2.start(); t1.join(); t2.join();
System.out.println(c.get());
\end{minted}    

\pause
Execution 1: \texttt{t1=1 t2=2 main=2}

\pause
Execution 2: \texttt{t1=2 t2=1 main=2}

\pause
Execution 3: \texttt{t1=2 t2=2 main=2}

\end{frame}

\begin{frame}[t,fragile]{Concurrent consistency}

\begin{itemize}
    \item Nothing crashes
    \pause
    \item When I run the program, it works as intended
    \pause
    \item All operations work ''logically''
\end{itemize}
\pause
Possible formalization: all operations could be treated as ''atomic'' (non-divisible, transactional) and ordered on a single timeline.
\end{frame}

\questiontime{
\begin{itemize}
    \item Concurrently consistent: all operations are ordered on a single timeline 
    \item Interleaving model: all executed  instructions are totally ordered
\end{itemize}
    Does it mean any concurrent data structure is consistent if we use interleaving model?
}

\begin{frame}[t,fragile,noframenumbering]{Concurrent consistency}

\begin{itemize}
    \item Nothing crashes    
    \item When I run the program, it works as intended
    \item All operations work ''logically''
\end{itemize}

Possible formalization: all operations (methods of corresponding concurrent data structure class) could be treated as ''atomic'' (non-divisible, transactional) and ordered on single timeline. 

\pause
There are other approaches, see consistency models in Lecture~\foundationsNum.

\end{frame}


\begin{frame}[t]{Toy problem: thread-safe counter}
\framesubtitle{How to implement?}

Our current requirements:
\begin{itemize}
    \item all events (method calls) could be ordered as if they executed sequentially
    \item in-thread events and operations are ''sequential'', but may be reordered up to ''synchronization points''
\end{itemize}

\pause
Synchronization points we know so far:
\begin{itemize}
    \item \texttt{Thread.start}
    \item \texttt{Thread.join}
\end{itemize}

\end{frame}

\questiontime{If you have only \texttt{Thread.start} and \texttt{Thread.join} as concurrent primitives, how would you implement thread-safe counter?}

\begin{frame}[t,noframenumbering]{Toy problem: thread-safe counter}
\framesubtitle{How to implement?}

Synchronization points we know so far:
\begin{itemize}
    \item \texttt{Thread.start}
    \item \texttt{Thread.join}
\end{itemize}

Looks like that is not enough.

\pause
When some thread executes \texttt{counter.increment}, other threads:
\begin{itemize}
        \item allowed to execute \texttt{counter.get}? \pause No, read-write data race!
        \pause
        \item allowed to execute \texttt{counter.increment}? \pause No, write-write data race!
\end{itemize}

\pause
Conclusion: 
\begin{itemize}
  \item avoid concurrent execution of the same code block by different threads (mutual exclusion)
  \pause
  \item guard instruction sequences against concurrent modification (code locking)
  \pause
  \item guarantee that only one thread may enter some code fragment (critical section)
\end{itemize}
\end{frame}

\section{Mutual exclusion}
\showTOCSub

\begin{frame}[fragile]{Mutual exclusion}
\framesubtitle{Naming}

Mutual exclusion: no more than one thread enters code fragment (critical section)

\begin{minted}{java}
interface Lock {
    void lock();
    void unlock();
}
\end{minted}

\pause

\begin{minted}{java}
interface Mutex {
    void enter();
    void exit();
}
\end{minted}

\pause

\begin{minted}{java}
interface CriticalSection {
    void begin();
    void end();
}
\end{minted}
\end{frame}


\begin{frame}[fragile]{Mutual exclusion}
\framesubtitle{Usage pattern}

Lock usage\footnote{\tiny\url{https://docs.oracle.com/en/java/javase/11/docs/api/java.base/java/util/concurrent/locks/Lock.html}}:

\begin{minted}{java}
  Lock lock = ...
  lock.lock();
  try {
    ...
  } finally {
    lock.unlock();
  }
\end{minted}

\pause

\begin{itemize}
    \item \textbf{If you are not using try-finally for locks -- you are writing incorrect code}
\end{itemize}
\end{frame}

\questiontime{Assume your program \texttt{lock}s in one method and \texttt{unlock}s in other. Which constructs for control flow could ''spoil'' your locking invariants?}

\begin{frame}[fragile]{Exceptions are hard}

{\tiny\url{https://github.com/Svazars/parallel-programming/blob/main/hw/block1/2.1/readme.markdown}}

\begin{homeworkmail}{Task \taskAsyncException.a}
    Is it possible that some exception would happen \textbf{inside} \texttt{lock} or \texttt{unlock} operation? Justify your answer by using precise \texttt{chapter.section} number from Java Language Specification.
}
\end{homeworkmail}

Help: $\sqrt[3]{1331}$ is good magic number.

\begin{homeworkmail}{Task \taskAsyncException.b}
    Is it possible to design ''bullet-proof'' (w.r.t. exceptions) concurrency primitives in Java language? Justify your answer by using precise JDK Enhancement Proposal number.
}
\end{homeworkmail}

Help: $\sqrt{72900}$ is good magic number, too.

\pause
\textbf{Warning:} these tasks are hard. It usually takes 3-5 attempts to ''defend'' your answer.

\end{frame}


\begin{frame}[fragile]{Mutex basics}

\begin{minted}{java}
interface Lock { 
    void lock(); 
    void unlock(); 
}
\end{minted}

\begin{itemize}
    \item Only one contending thread enters critical section. \textbf{Mutual exclusion}.
\end{itemize}

\pause What should other contending threads do?

\begin{itemize}
    \pause \item Await their ''turn''
\end{itemize}

\pause What exactly should current thread do when mutex is already busy?

\begin{itemize}
    \pause \item Release current scheduling quantum
\end{itemize}

\pause When thread will be awaken?

\begin{itemize}
    \pause \item randomly after some time period (if mutex is still busy, thread will be suspended again)
    \pause \item when mutex is unlocked (but mutex may become busy before thread is ''ready to go'')
    \pause \item ... 
\end{itemize}

\pause
Actually, you do not know. 
\pause 
Some time after other thread releases the mutex.
\end{frame}

\begin{frame}[t,fragile,noframenumbering]{Mutex basics}

\begin{minted}{java}
interface Lock { 
    void lock(); 
    void unlock(); 
}
\end{minted}

\begin{itemize}
    \item Only one contending thread enters critical section. \textbf{Mutual exclusion}.
    \pause \item Mutex affects thread scheduling.
\end{itemize}

\pause
\only<3>{Acquisition order}\only<4>{\textbf{Acquisition order}}, system throughput, observed latency depend on
\begin{itemize}
    \item Mutex implementation
    \item OS scheduling policy
    \item Non-determinism of CPU timings
\end{itemize}
\end{frame}

\begin{frame}[fragile]{Mutual exclusion solved with flags}
\framesubtitle{Thread A only}

\begin{lstlisting}
static boolean A_flag = false, B_flag = false;
static Lock l = ...;
void raise_X()        { l.lock(); try { X_flag = true;  } finally { l.unlock(); } }
void lower_X()        { l.lock(); try { X_flag = false; } finally { l.unlock(); } }
boolean is_raised_X() { l.lock(); try { return X_flag;  } finally { l.unlock(); } }
\end{lstlisting}

\begin{tabular}{p{7cm}p{7cm}}
    \begin{lstlisting}[
    linebackgroundwidth = 18 em,
    linebackgroundcolor={%      
      \btLstHL<2>{1}
      \btLstHL<3>{2}
      \btLstHL<4>{3}
      \btLstHL<5>{6}
      \btLstHL<6>{7}      
    }]
public void useful_A() {   
  raise_A();              // A.1 
  while (is_raised_B()) { // A.2
    continue;             // A.3      
  }
  critical_section();     // A.4
  lower_A();              // A.5
 }
    \end{lstlisting}
        
          &
    \begin{lstlisting}
public void useful_B() {   
  raise_B();              // B.1
  while (is_raised_A()) { // B.2    
    lower_B();            // B.3
    while (is_raised_A());// B.4
    raise_B();            // B.5
  }
  critical_section();     // B.6
  lower_B();              // B.7
 }
\end{lstlisting} \\
\end{tabular}
\end{frame}







\begin{frame}[fragile]{Mutual exclusion solved with flags}
\framesubtitle{Thread B only}

\begin{lstlisting}
static boolean A_flag = false, B_flag = false;
static Lock l = ...;
void raise_X()        { l.lock(); try { X_flag = true;  } finally { l.unlock(); } }
void lower_X()        { l.lock(); try { X_flag = false; } finally { l.unlock(); } }
boolean is_raised_X() { l.lock(); try { return X_flag;  } finally { l.unlock(); } }
\end{lstlisting}

\begin{tabular}{p{7cm}p{7cm}}
    \begin{lstlisting}
public void useful_A() {   
  raise_A();              // A.1 
  while (is_raised_B()) { // A.2
    continue;             // A.3      
  }
  critical_section();     // A.4
  lower_A();              // A.5
 }
    \end{lstlisting}
        
          &
    \begin{lstlisting}[
    linebackgroundwidth = 18 em,
    linebackgroundcolor={%      
      \btLstHLG<2>{1}
      \btLstHLG<3>{2}
      \btLstHLG<4>{3}
      \btLstHLG<5>{8}
      \btLstHLG<6>{9}      
    }]
public void useful_B() {   
  raise_B();              // B.1
  while (is_raised_A()) { // B.2    
    lower_B();            // B.3
    while (is_raised_A());// B.4
    raise_B();            // B.5
  }
  critical_section();     // B.6
  lower_B();              // B.7
 }
\end{lstlisting} \\
\end{tabular}

\end{frame}






\begin{frame}[fragile]{Mutual exclusion solved with flags}
\framesubtitle{Contention}

\begin{lstlisting}
static boolean A_flag = false, B_flag = false;
static Lock l = ...;
void raise_X()        { l.lock(); try { X_flag = true;  } finally { l.unlock(); } }
void lower_X()        { l.lock(); try { X_flag = false; } finally { l.unlock(); } }
boolean is_raised_X() { l.lock(); try { return X_flag;  } finally { l.unlock(); } }
\end{lstlisting}

\begin{tabular}{p{7cm}p{7cm}}
    \begin{lstlisting}[
    linebackgroundwidth = 16 em,
    linebackgroundcolor={%
      \btLstHL<2-3>{1}
      \btLstHL<4-5>{2}
      \btLstHL<6>{3}
      \btLstHL<7>{4}
      \btLstHL<8>{3}
      \btLstHL<9-12>{4}
      \btLstHL<13>{3}
      \btLstHL<14>{6}
      \btLstHL<15>{7}
    }]
public void useful_A() {   
  raise_A();              // A.1 
  while (is_raised_B()) { // A.2
    continue;             // A.3      
  }
  critical_section();     // A.4
  lower_A();              // A.5
 }
    \end{lstlisting}
        
          &
    \begin{lstlisting}[
    linebackgroundwidth = 16 em,
    linebackgroundcolor={%
      \btLstHLG<3-4>{1}
      \btLstHLG<5-9>{2}
      \btLstHLG<10>{3}
      \btLstHLG<11>{4}
      \btLstHLG<12-16>{5}
      \btLstHLG<17>{6}
      \btLstHLG<18>{3}
      \btLstHLG<19>{8}
    }]
public void useful_B() {   
  raise_B();              // B.1
  while (is_raised_A()) { // B.2    
    lower_B();            // B.3
    while (is_raised_A());// B.4
    raise_B();            // B.5
  }
  critical_section();     // B.6
  lower_B();              // B.7
 }
\end{lstlisting} \\
\end{tabular}

\end{frame}


\begin{frame}[fragile]{Mutual exclusion and deadlock-freedom}

{\tiny\url{https://github.com/Svazars/parallel-programming/blob/main/hw/block1/2.2/readme.markdown}}

\begin{homeworkmail}{Task \taskProofMutex}
    Prove that algorithm on previous slide guarantees mutual exclusion for 2 threads.
    Assume it is not and get a contradiction.
}
\end{homeworkmail}

\begin{homeworkmail}{Task \taskProofDeadlockFreedom}
    Prove that algorithm on previous slide is free of deadlocks.   
}
\end{homeworkmail}

\textbf{Suggested reading}: use companion slides for ''Herlihy, Shavit: The Art of Multiprocessor Programming''\footnote{\url{https://booksite.elsevier.com/9780123973375}}, Lecture slides, Chapter 01, slides 43-72.

\end{frame}


\begin{frame}[t,fragile,noframenumbering]{Mutex basics}

\begin{minted}{java}
interface Lock { 
    void lock(); 
    void unlock(); 
}
\end{minted}

\begin{itemize}
    \item Only one contending thread enters critical section. \textbf{Mutual exclusion}.
    \item Mutex affects thread scheduling.    
    \item At least one contending thread enters critical section. \textbf{Deadlock-freedom}.
\end{itemize}

\end{frame}



\begin{frame}[fragile]{Mutual exclusion solved with flags}
\framesubtitle{Starvation}

\only<23->  {\textbf{Starvation}: user of concurrent object \textit{could} be delayed for \textit{arbitrary} time if there are other users of the same object.} \only<24->  {\textbf{Unfair mutex}: current thread starves, whole system progresses. }

\begin{tabular}{p{7cm}p{7cm}}
    \begin{lstlisting}[
    linebackgroundwidth = 16 em,
    linebackgroundcolor={%
      \btLstHL<2-3>{1}
      \btLstHL<4-5>{2}
      \btLstHL<6-9>{3}
      \btLstHL<10>{6}
      \btLstHL<11>{7}
      \btLstHL<12-13>{1}
      \btLstHL<14>{2}
      \btLstHL<15-18>{3}
      \btLstHL<19>{6}
      \btLstHL<20>{7}
    }]
public void useful_A() {   
  raise_A();              // A.1 
  while (is_raised_B()) { // A.2
    continue;             // A.3      
  }
  critical_section();     // A.4
  lower_A();              // A.5
 }
    \end{lstlisting}
        
          &
    \begin{lstlisting}[
    linebackgroundwidth = 16 em,
    linebackgroundcolor={%
      \btLstHLG<3-4>{1}
      \btLstHLG<5-6>{2}
      \btLstHLG<7>{3}
      \btLstHLG<8>{4}
      \btLstHLG<9-12>{5}
      \btLstHLG<13-15>{6}
      \btLstHLG<16>{3}
      \btLstHLG<17>{4}
      \btLstHLG<18-20>{5}
    }]
public void useful_B() {   
  raise_B();              // B.1
  while (is_raised_A()) { // B.2    
    lower_B();            // B.3
    while (is_raised_A());// B.4
    raise_B();            // B.5
  }
  critical_section();     // B.6
  lower_B();              // B.7
 }
\end{lstlisting} \\
\end{tabular}

\only<1-9>  {\texttt{Thread A acquisitions: 0, Thread B acquisitions: 0}}
\only<10-18>{\texttt{Thread A acquisitions: 1, Thread B acquisitions: 0}}
\only<19-21>{\texttt{Thread A acquisitions: 2, Thread B acquisitions: 0}}
\only<22>   {\texttt{Thread A acquisitions: N, Thread B acquisitions: 0}}

\end{frame}

\questiontime{What is the difference between \textbf{deadlock-freedom} and \textbf{starvation-freedom}?}


% \begin{frame}[fragile]{Livelock}
% 
% TODO: example two whiles try-and-retreat
% 
% \end{frame}

\begin{frame}[t,fragile,noframenumbering]{Mutex basics}

\begin{minted}{java}
interface Lock { 
    void lock(); 
    void unlock(); 
}
\end{minted}

\begin{itemize}
    \item Only one contending thread enters critical section. \textbf{Mutual exclusion}.
    \item Mutex affects thread scheduling.    
    \item At least one contending thread enters critical section. \textbf{Deadlock-freedom}.
    \item Some contending threads could lag. \textbf{No starvation-freedom/fairness by default}.
\end{itemize}

\end{frame}


\begin{frame}[t,fragile]{Check-then-act}

\begin{minted}{java}
class ThreadSafeContainer {
    List a = new ArrayList<>();
    Lock l = ... ;
    public void add(Object o) {
        l.lock(); try { a.add(o); } finally { l.unlock(); }}
    public boolean contains(Object o) {
        l.lock(); try { return a.contains(o); } finally { l.unlock(); }}
    public void addIfAbsent(Object o) {
        if (!contains(o)) add(o); }
}
\end{minted}
\pause No data race. \pause Inconsistent behaviour due to race condition.

\pause \textbf{Remember: state of concurrent system may have changed since your last inspection}
\end{frame}

% \begin{frame}[fragile]{Check-then-act + retry == livelock}
% TODO: maybe homework?
% \end{frame}


\begin{frame}[t,fragile,noframenumbering]{Mutex basics}
\framesubtitle{Summary}

\begin{minted}{java}
interface Lock { 
    void lock(); 
    void unlock(); 
}
\end{minted}

\begin{itemize}
    \item Only one contending thread enters critical section. \textbf{Mutual exclusion}.
    \item Mutex affects thread scheduling.    
    \item At least one contending thread enters critical section. \textbf{Deadlock-freedom}.
    \item Some contending threads could lag. \textbf{No starvation-freedom/fairness by default}.
    \item Mutex helps to avoid data races, \textbf{does not} magically solves all race conditions.
\end{itemize}

\end{frame}


\questiontime{How to fix \texttt{addIfAbsent(e) = if (!contains(e)) add(e);}?}
\questiontime{How to fix \texttt{addIfAbsent(e) = if (!contains(e)) add(e);}? 

It it OK to grab the lock in \texttt{addIfAbsent} and then again in \texttt{contains}?}

\section{Mutex}
\subsection{Reentrancy}
\showTOCSub

\begin{frame}[t, fragile]{NonReentrantLock}

\texttt{NonReentrantLock = \{ boolean busy \}}

\pause

Single-threaded deadlock №1:
\begin{minted}{java}
    void foo() { l.lock(); l.lock(); }
\end{minted}

\pause

Single-threaded deadlock №2:

\begin{minted}{java}
    void add(Object o) {
        l.lock(); try { a.add(o); } finally { l.unlock(); }}
    boolean contains(Object o) {
        l.lock(); try { return a.contains(o); } finally { l.unlock(); }}
    void addIfAbsent(Object o) {
        l.lock(); try { if (!contains(o)) add(o); } finally { l.unlock(); }}
\end{minted}
\end{frame}


\begin{frame}[t,fragile]{ReentrantLock}

\begin{itemize}
    \item ReentrantLock, ReentrantMutex
    \item RecursiveLock, RecursiveMutex
\end{itemize}

\pause

\texttt{ReentrantLock = \{ Owner owner, int count \}}

Not every \texttt{unlock} actually releases ownership

\pause
Important concepts:
\begin{itemize}
    \item Structured locking: every \texttt{lock} paired with \texttt{unlock}
    \item Ownership: unique Thread ID (\texttt{Thread.currentThread()}\footnote<3->{\tiny\url{https://docs.oracle.com/en/java/javase/11/docs/api/java.base/java/lang/Thread.html#currentThread()}}) to distinguish owners
\end{itemize}

\pause

{\tiny\url{https://github.com/Svazars/parallel-programming/blob/main/hw/block1/2.3/readme.markdown}}
\begin{homeworkmail}{Task \taskReentrant}
    Implement reentrant mutex using non-reentrant one.
}
\end{homeworkmail}
\end{frame}

\questiontime{Concurrency is hard! Why would anybody use NonReentrantMutex?}


\begin{frame}[fragile]{Toy problem: thread-safe counter}
\framesubtitle{ReentrantLock-based\footnote{\tiny\url{https://docs.oracle.com/en/java/javase/11/docs/api/java.base/java/util/concurrent/locks/ReentrantLock.html}} implementation}

\begin{minted}{java}
public class Counter {
    private final Lock lock = new ReentrantLock();
    private long counter;
    public Counter(long initial) { counter = initial; }
    public void increment() {
        lock.lock(); try { counter++; } finally { lock.unlock(); }
    }
    public long get() { 
        lock.lock(); try {  return counter; } finally { lock.unlock(); }
    }
}
\end{minted}
\end{frame}


\begin{frame}[fragile,noframenumbering]{Toy problem: thread-safe counter}
\framesubtitle{ReentrantLock-based implementation}

\begin{minted}{java}
public class Counter {
    private final Lock lock = new ReentrantLock();
    private long counter;
    public Counter(long initial) { counter = initial; }
    public void increment() {
        lock.lock(); try { counter++; } finally { lock.unlock(); }
    }
    public long get() { 
        lock.lock(); try {  return counter; } finally { lock.unlock(); }
    }
}
\end{minted}

\begin{itemize}
    \item \pause Data races? \pause Race conditions? \pause Concurrently consistent?
    \item \pause Deadlock-freedom? \pause Starvation-freedom? \pause Scalability?
\end{itemize}
\end{frame}


\begin{frame}[fragile]{Homework: thread-safe counters}

{\tiny\url{https://github.com/Svazars/parallel-programming/blob/main/hw/block1/2.4/readme.markdown}}

\begin{homeworkmail}{Task \taskCounters}
    \begin{itemize}
        \item Implement different kinds of thread-safe counters
        \item Analyze scalability using JMH\footnote{\tiny\url{https://github.com/openjdk/jmh}}
        \item Find inconsistencies in ''highly-distributed'' counters implementations
    \end{itemize}
}
\end{homeworkmail}

\end{frame}


% \begin{frame}{Toy problem: thread-safe counter}
% \framesubtitle{Mutex implementation: brief discussion}
% 
% Any concurrent algorithm should be analyzed for the \textbf{key} properties:
% \begin{itemize}
%     \item Safety (correctness)
%     \begin{itemize}
%         \item Implements contract (consistency)
%         \item Absence of invariant violations
%         \item Absence of data races
%         \item Absence of concurrent logical errors (unfortunate race conditions)
%     \end{itemize}
%     \item Liveness (progress)
%     \begin{itemize}
%         \item Deadlock-freedom
%         \item Livelock-freedom    
%         \item Starvation-freedom
%     \end{itemize}
%     \item Performance
%     \begin{itemize}
%         \item Throughput (fast-path/slow-path overheads)
%         \item Latency (fairness, priority inversion)
%         \item Scalability
%     \end{itemize}
% \end{itemize}
% \end{frame}

\subsection{Admission policy}
\showTOCSub


\begin{frame}[fragile]{Fairness}

\begin{itemize}
    \item \textbf{Mutual exclusion}: no more than one thread enters
    \item \textbf{Deadlock-freedom}: some thread eventually enters
    \item \textbf{Starvation-freedom}: this thread eventually enters
\end{itemize}

\pause

If there are \texttt{N} contending threads, what is ''distribution of enters''?

\begin{itemize}
    \pause \item Arbitrary
    \pause \item Priority-based
    \pause \item Even    
\end{itemize}

\pause
We could empirically measure this: {\url{https://en.wikipedia.org/wiki/Fairness_measure}}

\pause
In our course we will use ''all-or-nothing'' approach: thread \textbf{could starve} or \textbf{never starves}.
\end{frame}


\questiontime{assume some thread starves. Does it mean that throughput of the whole program will be low?}

\begin{frame}[t,fragile]{Fairness and performance}
\framesubtitle{LIFO}

\only<1-5>{ \texttt{\ \ \ \ \ \ \ \ \ A work: 0  \ \ \ \ \ \ \ \ \ \ \ \ \ \ \ \ \ \ \ \ \ \ \ \ \ \ \ \     B work 0}}
\only<6-8>{ \texttt{\ \ \ \ \ \ \ \ \ A work: 1  \ \ \ \ \ \ \ \ \ \ \ \ \ \ \ \ \ \ \ \ \ \ \ \ \ \ \ \     B work 0}}
\only<9-11>{\texttt{\ \ \ \ \ \ \ \ \ A work: 2  \ \ \ \ \ \ \ \ \ \ \ \ \ \ \ \ \ \ \ \ \ \ \ \ \ \ \ \     B work 0}}
\only<12->{ \texttt{\ \ \ \ \ \ \ \ \ A work: 3  \ \ \ \ \ \ \ \ \ \ \ \ \ \ \ \ \ \ \ \ \ \ \ \ \ \ \ \     B work 0}}


\begin{tabular}{p{7cm}p{7cm}}
    \begin{lstlisting}[
    linebackgroundwidth = 16 em,
    linebackgroundcolor={%
      \btLstHL<2-4>{2}
      \btLstHL<5>{3}
      \btLstHL<6>{4}
      \btLstHL<7>{5}
      \btLstHL<8>{3}
      \btLstHL<9>{4}
      \btLstHL<10>{5}
      \btLstHL<11>{3}
      \btLstHL<12>{4}
      \btLstHL<13->{5}
    }]
public void threadA() {
  while (true) {
    unfairMutex.lock();
    doWork();
    unfairMutex.unlock();  
  }
 }
    \end{lstlisting}
        
          &
    \begin{lstlisting}[
    linebackgroundwidth = 16 em,
    linebackgroundcolor={%
      \btLstHLG<3>{2}
      \btLstHLG<4->{3}
    }]
public void threadB() {   
  while (true) {
    unfairMutex.lock();
    doWork();
    unfairMutex.unlock(); 
  }
 }
\end{lstlisting} \\
\end{tabular}

\only<1-4>{\texttt{\ \ \ A context switch: 0   \ \ \ \ \ \ \ \ \ \ \ \ \ \ \ \ \ \      B context switch 0}}
\only<5->{\texttt{\ \ \ A context switch: 0   \ \ \ \ \ \ \ \ \ \ \ \ \ \ \ \ \ \      B context switch 1}}

\only<14->{
    \begin{itemize}
        \item Unfair locks allow some threads to better utilize scheduling quantum. Better throughput.
        \item Unfair locks could cause starvation. Higher latency.
    \end{itemize}
}
\end{frame}


\begin{frame}[t,fragile]{Fairness and performance}
\framesubtitle{FIFO}

\only<1-5>{ \texttt{\ \ \ \ \ \ \ \ \ A work: 0  \ \ \ \ \ \ \ \ \ \ \ \ \ \ \ \ \ \ \ \ \ \ \ \ \ \ \ \     B work 0}}
\only<6-8>{ \texttt{\ \ \ \ \ \ \ \ \ A work: 1  \ \ \ \ \ \ \ \ \ \ \ \ \ \ \ \ \ \ \ \ \ \ \ \ \ \ \ \     B work 0}}
\only<9-11>{\texttt{\ \ \ \ \ \ \ \ \ A work: 1  \ \ \ \ \ \ \ \ \ \ \ \ \ \ \ \ \ \ \ \ \ \ \ \ \ \ \ \     B work 1}}
\only<12-14>{\texttt{\ \ \ \ \ \ \ \ \ A work: 2  \ \ \ \ \ \ \ \ \ \ \ \ \ \ \ \ \ \ \ \ \ \ \ \ \ \ \ \     B work 1}}
\only<15->{\texttt{\ \ \ \ \ \ \ \ \ A work: 2  \ \ \ \ \ \ \ \ \ \ \ \ \ \ \ \ \ \ \ \ \ \ \ \ \ \ \ \     B work 2}}


\begin{tabular}{p{7cm}p{7cm}}
    \begin{lstlisting}[
    linebackgroundwidth = 16 em,
    linebackgroundcolor={%
      \btLstHL<2-3>{2}
      \btLstHL<4-5>{3}
      \btLstHL<6>{4}
      \btLstHL<7>{5}
      \btLstHL<8-11>{3}
      \btLstHL<12>{4}
      \btLstHL<13>{5}
      \btLstHL<14->{3}
    }]
public void threadA() {
  while (true) {
    fairMutex.lock();
    doWork();
    fairMutex.unlock();  
  }
 }
    \end{lstlisting}
        
          &
    \begin{lstlisting}[
    linebackgroundwidth = 16 em,
    linebackgroundcolor={%
      \btLstHLG<3-4>{2}
      \btLstHLG<5-8>{3}
      \btLstHLG<9>{4}
      \btLstHLG<10>{5}
      \btLstHLG<11-14>{3}
      \btLstHLG<15>{4}
      \btLstHLG<16>{5}
      \btLstHLG<17->{3}
    }]
public void threadB() {   
  while (true) {
    fairMutex.lock();
    doWork();
    fairMutex.unlock(); 
  }
 }
\end{lstlisting} \\
\end{tabular}

\only<1-4>{\texttt{\ \ \ A context switch: 0   \ \ \ \ \ \ \ \ \ \ \ \ \ \ \ \ \ \      B context switch 0}}
\only<5-7>{\texttt{\ \ \ A context switch: 0   \ \ \ \ \ \ \ \ \ \ \ \ \ \ \ \ \ \      B context switch 1}}
\only<8-10>{\texttt{\ \ \ A context switch: 1   \ \ \ \ \ \ \ \ \ \ \ \ \ \ \ \ \ \      B context switch 1}}
\only<11-13>{\texttt{\ \ \ A context switch: 1   \ \ \ \ \ \ \ \ \ \ \ \ \ \ \ \ \ \      B context switch 2}}
\only<14-16>{\texttt{\ \ \ A context switch: 2   \ \ \ \ \ \ \ \ \ \ \ \ \ \ \ \ \ \      B context switch 2}}
\only<17->{\texttt{\ \ \ A context switch: 2   \ \ \ \ \ \ \ \ \ \ \ \ \ \ \ \ \ \      B context switch 3}}

\only<18->{
    \begin{itemize}
        \item Fair locks could trigger many context switches. Lower utilization.
        \item Fair locks encourage better responsiveness. Lower latency.
        \item Fair locks provide more guarantees on lock ordering. Better predictability.
    \end{itemize}
}
\end{frame}

%\begin{frame}{High-level mutex design choices}
%
%Contended mutex: 
%\begin{itemize}
%    \item single \textbf{owner}
%    \item set of \textbf{waiters} (EnterSet)
%    \item set of \textbf{arriving threads} (ArriveSet)
%\end{itemize}
%
%TODO: room based picture
%
%\end{frame}
%
%
%\begin{frame}{Admission policy}
%\framesubtitle{Contenders ordering}
%
%Contended mutex: 
%\begin{itemize}
%    \item single \textbf{owner}
%    \item set of \textbf{waiters} (EnterSet)
%    \item set of \textbf{arriving threads} (ArriveSet)
%\end{itemize}
%
%TODO: room based picture
%
%\pause
%
%How to manage EnterSet?
%
%\pause
%
%\begin{itemize}
%    \item last-in-first-out (LIFO)
%    \item first-in-first-out (FIFO)
%    \item priority queue or random choice    
%\end{itemize}
%
%\pause
%
%How to manage ArriveSet?
%
%\pause
%
%\begin{itemize}
%    \item try-lock-then-wait (ArriveSet > EnterSet)
%    \item if-busy-then-wait (ArriveSet < EnterSet)
%    \item random or heuristic choice    
%\end{itemize}
%
%\pause
%
%Race conditions everywhere!
%\end{frame}
%
%\begin{frame}[t]{Admission policy}
%\framesubtitle{Design space}
%
%Depends on your goal:
%\begin{itemize}
%    \item Max throughput: LIFO (unfair, best average case, degraded outliers)
%\end{itemize}    
%
%\end{frame}
%
%\questiontime{Why LIFO admission policy for mutex provides the best throughput?}

\begin{frame}[t,noframenumbering]{Admission policy}
\framesubtitle{Design space}

Depends on your goal:
\begin{itemize}
    \pause \item Max throughput: \pause LIFO (unfair, best average case, degraded outliers)
    \pause \item Latency: \pause FIFO (fair, guaranteed worst case)
    \pause \item Predictability: \pause almost FIFO or priorities (semi-fair, acceptable worst case)
\end{itemize}

\pause

Everything has negative side:
\begin{itemize}
    \item Starvation, Priority inversion, Throughput, Deadlock probability
\end{itemize}

\pause

\textbf{Your concurrent data structures should document admission policy and starvation scenarios for blocking methods}

\end{frame}

\begin{frame}[t,fragile,noframenumbering]{Admission policy}
\framesubtitle{java.util.concurrent.ReentrantLock}

\textbf{Your concurrent data structures should document admission policy and starvation scenarios for blocking methods}

\pause

\begin{minted}{java}
    ReentrantLock(boolean fair)
\end{minted}

\pause

\begin{quote}
When set true, under contention, locks favor granting access to the longest-waiting thread. Otherwise this lock does not guarantee any particular access order. 
\end{quote}

\pause

\begin{quote}
Programs using fair locks accessed by many threads may display lower overall throughput (i.e., are slower; often much slower) than those using the default setting, but have smaller variances in times to obtain locks and guarantee lack of starvation.
\end{quote}

\pause
\begin{quote}
Note however, that fairness of locks does not guarantee fairness of thread scheduling ... Also note that the untimed tryLock() method does not honor the fairness setting. 
\end{quote}

\pause
Task~\taskCounters, Easy level: {\tiny\url{https://github.com/Svazars/parallel-programming/blob/main/hw/block1/2.4/readme.markdown}}

% TODO: dave dice on admision policy
% https://web.archive.org/web/20160316200659/https://blogs.oracle.com/dave/entry/locks_with_lifo_admission_order

\end{frame}


\subsection{Visibility}
\showTOCSub

\begin{frame}{Visibility and consistency}

\begin{itemize}
    \item all \textt{lock} and \texttt{unlock} operations of \textbf{particular mutex} are totally ordered
    \item intra-thread \textt{lock} and \texttt{unlock} operations of \textbf{all mutexes} are totally ordered
\end{itemize}

\pause
Partial orders are tricky\footnote<2->{\url{https://en.wikipedia.org/wiki/Partially_ordered_set}}

\pause
Synchronization points you know so far:
\begin{itemize}
    \item \texttt{Thread.start}
    \item \texttt{Thread.join}
    \item \texttt{Lock.lock}
    \item \texttt{Lock.unlock}
\end{itemize}
\end{frame}

\questiontime{It would be \textbf{much} easier to say that all critical sections (code between \texttt{lock} and \texttt{unlock}) of \textbf{all} mutexes have a strict total order.

Why do we use much weaker partial ordering?
}

\begin{frame}[fragile]{Visibility and consistency}
\framesubtitle{Insufficient ordering}

\begin{minted}{java}
static int x, y;
void threadA() {
    lock.lock(); try { x = 1; y = 1; } finally { lock.unlock(); }
}
void threadB() {
    lock.lock(); try { x = 2; y = 2; } finally { lock.unlock(); }
}
void threadC() {
    System.out.println(x);
    System.out.println(y);
}
\end{minted}

\pause

Possible result: \texttt{x=2 y=0}

\end{frame}

\begin{frame}{Mutual exclusion}
\framesubtitle{Conclusion}

\begin{itemize}
    \item Mutual exclusion maintains ''order of execution'' for code fragment, one thread a time
    \item Implicit control flow (e.g. exceptions) may violate consistency of concurrent primitive
    \item Performance depends on OS (scheduling quantum, scheduling policy, context switch overheads, priority) and particular implementation (admission policy)
    \item There are different flavours of locking primitives (reentrancy, fairness)
\end{itemize}

Locks help to solve some problems:
\begin{itemize}
    \item avoid data race
    \item prevent race condition
    \item implement thread-safety
\end{itemize}
but may introduce new challenges:
\begin{itemize}
    \item deadlock
    \item starvation/unfairness
    \item sequential part of execution (see Amdahl's law in Lecture~\introNum)
\end{itemize}
\end{frame}


\section{Patterns}
\subsection{Code locking}
\showTOCSub

\begin{frame}[fragile]{Code locking}

\begin{minted}{java}
    enum Grade { A, B, C, FAIL }
    static long grades[] = new long[Grade.values().length()];
    public static void gradeStudent(Grade g) {
        grades[g.ordinal()]++;
    }
\end{minted}

How to make \texttt{gradeStudent} thread-safe?

\pause
\begin{minted}{java}
    static Lock lock = new ReentrantLock();
    public static void gradeStudent(Grade g) {
        lock.lock(); 
        try { 
            grades[g.ordinal()]++; 
        } finally { 
            lock.unlock(); 
        }
    }
\end{minted}
\end{frame}

\subsection{Data locking}

\begin{frame}[fragile]{Data locking}

\begin{minted}{java}
    public static void gradeStudent(Grade g) {
        lock.lock(); 
        try {
            grades[g.ordinal()]++; 
        } finally { 
            lock.unlock(); 
        }
    }
\end{minted}

How to make program more scalable?

\pause

\begin{minted}{java}
    static Counter[] grades = new ThreadSafeCounter[Grade.values().length];
    public static void gradeStudent(Grade g) {
        grades[g.ordinal()].increment();
    }
\end{minted}

\end{frame}

%\questiontime{In Java, every method is attached to the object instance which is actually data. Should not we always use ''Data locking'' terminology?}

%\questiontime{In Java, every method is written in bytecode so effectively we are guarding some code fragment. Should not we always use ''Code locking'' terminology?}

\subsection{Lock splitting}

\begin{frame}[fragile]{Lock splitting}

\begin{minted}{java}
    static int[] passedExams = new int[StudentList.size()]; // millions!
    static Lock lock = new Lock();
    public static void pass(Student s) {
        lock.lock();
        try { 
            passedExams[s.number()]++; 
        } finally { 
            lock.unlock(); 
        }
    }
\end{minted}

\pause
Assume we cannot afford to allocate millions of \texttt{ThreadSafeCounter} instances.

\pause
Divide-and-conquer using arbitrary granularity.

\end{frame}

\begin{frame}[fragile,noframenumbering]{Lock splitting}

\begin{minted}{java}
    static int[] passedExams = new int[StudentList.size()]; // millions!
    static Lock[] locks = new Lock[1 + (passedExams.length / 1_000)];
    public static void pass(Student s) {
        int sNum = s.number();
        int lockNum = sNum / 1_000;
        Lock lock = locks[lockNum];
        lock.lock();
        try { 
            passedExams[sNum]++; 
        } finally { 
            lock.unlock(); 
        }}
\end{minted}

\pause
Task~\taskCounters, Medium level: {\tiny\url{https://github.com/Svazars/parallel-programming/blob/main/hw/block1/2.4/readme.markdown}}

\end{frame}

\section{Bug prevention}
\showTOCSub

\begin{frame}[fragile]{Inevitable evil}
\framesubtitle{''I would never do it''}

\begin{minted}{java}
threadA() {
    l1.lock();
    try {
        l2.lock();
        try { ... } finally { l2.unlock(); }
    } finally { l1.unlock(); }
}
threadB() {
    l2.lock();
    try {
        l1.lock();
        try { ... } finally { l1.unlock(); }
    } finally { l2.unlock(); }
}
\end{minted}
\end{frame}

\begin{frame}[fragile]{Inevitable evil}
\framesubtitle{''Oops!... I did it again''}

\begin{minted}{java}
void transfer(long sum, Account a, Account b) {
    a.lock.lock();
    try {
        b.lock.lock();
        try {
            if (a.withdraw(sum)) {
                b.add(sum)
            }
        } finally { b.lock.unlock(); }
    } finally { a.lock.unlock(); }
}
\end{minted}

\pause

\begin{minted}{java}
threadA() { transfer(1, A, B); }
threadB() { transfer(1, B, A); }
\end{minted}
\end{frame}


\begin{frame}{Deadlock prevention}

\textbf{Ultimate deadlock prevention weapon}

\pause

Do not use blocking methods ;)

\end{frame}

\begin{frame}[t,noframenumbering]{Deadlock prevention}
\framesubtitle{Wishful thinking}

Minimize attack surface:
\begin{itemize}    
    \item Use single lock in the program
    \pause
    \item Use recursive locks
    \pause
    \item Use single thread in the program    
    \pause
    \item Use message passing (copy and transfer) instead of shared mutable state
    \pause
    \item Use high-level abstractions (\texttt{stream.parallel.map.collect}) instead of low-level ones (\texttt{mutex.lock/unlock})
\end{itemize}

\end{frame}

\begin{frame}[t,noframenumbering]{Deadlock prevention}
\framesubtitle{Practice}

Minimize attack surface:
\begin{itemize}
    \item Use recursive locks
    \item Use high-level abstractions and thread-safe classes    
\end{itemize}

\end{frame}

\questiontime{You have \texttt{private Lock} instance and public \texttt{foo} method. How should you use lock inside method to avoid deadlocks on this instance?}

\begin{frame}[t,noframenumbering]{Deadlock prevention}

Minimizing attack surface:
\begin{itemize}
    \item Use high-level abstractions and thread-safe classes
    \item Use recursive locks
    \item Do not publish internal locks
    \item Avoid blocking calls inside critical section (use ''leaf'' locking)
\end{itemize}

\pause

Specialized techniques:
\begin{itemize}
    \item Lock ordering (\texttt{lockA} < \texttt{lockB}). First sort, then lock!
    \pause \item Locking hierarchies (\texttt{lockA.num = 1}, \texttt{lockB.num = 2}). Always lock greater numbers!
\end{itemize}
\pause Requires thinking at design time.

\pause
{\tiny\url{https://github.com/Svazars/parallel-programming/blob/main/hw/block1/2.5/readme.markdown}}
\begin{homeworkmail}{Task \taskEmpireLock}{
    Open \url{https://deadlockempire.github.io}, pass all ''Locks'' levels.
}
\end{homeworkmail}
\end{frame}


\begin{frame}[t,fragile]{Dining philosophers problem}

\begin{tikzpicture}[remember picture,overlay]
\node[xshift=4cm,yshift=-8cm] at (current page.center) {\includegraphics[width=0.35\textwidth]{./pics/Dining_philosophers_diagram.jpg}};
\end{tikzpicture}

{\tiny\url{https://github.com/Svazars/parallel-programming/blob/main/hw/block1/2.6/readme.md}}

\begin{homeworkcode}{Task~\taskCodeDining}
    Help them or they will starve to death!
    \begin{itemize}
        \item Three levels of difficulty
        \item Required time grows exponentially
    \end{itemize}
\end{homeworkcode}

\end{frame}



%\questiontime{Locking hierarchy reports deadlock in run-time, on erroneous attempt to ''get lower lock''. Actually it preliminary crashes the program, even if deadlock was not supposed to happen. Why do we call it ''deadlock prevention'' software development technique?}

%\begin{frame}{Lock convoy}
%
%
%First thread arrives to mutex. \pause Owns it. \pause
%
%Second thread arrives to mutex. \pause Waits for it. Quantum lost. \pause
%
%First thread leaves mutex. \pause Second thread gets ownership. \pause
%
%Third thread arrives to mutex. \pause Waits for it. Quantum lost. \pause
%
%First thread arrives to mutex. \pause Waits for it. Quantum lost. \pause
%
%Second thread leaves mutex ...
%
%\end{frame}

\newcommand{\tmpheader}{
\begin{itemize}
    \item When you call some code, it could acquire/release arbitrary locks
    \item When your code is invoked by some thread, that thread could already own arbitrary locks    
\end{itemize}
}

\begin{frame}[t]{Design challenges}

\tmpheader

\pause

Composability hell.

\end{frame}

\begin{frame}[t]{Design challenges}

\tmpheader

Trust no one
\begin{itemize}
    \item Before calling external code, release all locks
    \item Avoid using external locks
    \item Do not expose internal locks
    \item Start computation in special ''clean'' thread
\end{itemize}

\pause

But be friendly
\begin{itemize}
    \item Document locking policy inside class
    \item Document locking policy for users
\end{itemize}

\end{frame}
%%%\input{parts/signalling.tex}

\section{Summary}

\begin{frame}{Summary}

\begin{itemize}
    \item Mutual exclusion helps to achieve thread-safety
    \item \texttt{Mutex} (\texttt{lock}, \texttt{critical section}) provides easy-to-use and simple API. Key concepts:
    \begin{itemize}
        \item deadlock-freedom, starvation-freedom, reentrancy, admission policy, fairness
    \end{itemize} 

    \item There are different ways to structure concurrent programs with locks:
    \begin{itemize}
        \item code locking, data locking, lock splitting
    \end{itemize}

    \item Mutex is a blocking primitive, so be aware of possible deadlocks:
    \begin{itemize}
        \item recursive locks, encapsulation, enforcing lock acquisition order, avoiding external code invocation inside critical section
    \end{itemize}

    \item Documenting locking policy is a key to modular and reliable concurrent software
    \item Do not forget to read documentation of thread-safe classes you use
\end{itemize}

\end{frame}


%\begin{frame}{Summary}
%
%Some data races and race conditions could be avoided by mutual exclusion.
%
%\texttt{Mutex} (a.k.a. \texttt{lock}, a.k.a. \texttt{critical section}) provides easy-to-use and simple API.
%
%Do not forget to keep an eye on
%\begin{itemize}
%    \item safety/correctness, liveness/progress guarantees, visibility/consistency, performance
%\end{itemize}
%
%Use suitable locking policies for your use-cases
%\begin{itemize}
%    \item code locking, data locking, lock splitting
%\end{itemize}
%
%Remember to document locking policy to keep your program modular/reusable.
%
%Do not forget to read documentation of thread-safe classes you use.
%
%Condition variable allows to replace polling with OS-level signalling.
%
%Signalling protocols must be aware of
%\begin{itemize}
%    \item lost signals, predicate invalidation, spurious wakeups, fairness
%\end{itemize}
%\end{frame}

\begin{frame}{Summary: homework}

\texttt{Lecture 2, Tasks 2.*:} {\tiny\url{https://github.com/Svazars/parallel-programming/blob/main/hw/block1}}

\end{frame}


%Big homework (4w)
    %bounded thread pool, blocking awaits are self-helping,
%- design
%- tests
%- deadlock conditions
%- example of deadlock with resource-constrained cause


% \begin{frame}{Producer consumer}
% Unbounded queue with mutex
% \end{frame}
% 
% \begin{frame}{Publisher-subsrciber}
% Pull/push approaches, reactive systems
% \end{frame}
% 
% \begin{frame}{Backoff policies}
% Livelock
% \end{frame}
% 
% \begin{frame}{Problems}
% 
% message passing + backpressure 
% self impl thread-safe counter
% self impl striped lock (hashmap?)
% \end{frame}


\end{document}
