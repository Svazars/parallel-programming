\begin{frame}[fragile]{Mutual exclusion solved with flags}
\framesubtitle{Thread A only}

\begin{lstlisting}
static boolean A_flag = false, B_flag = false;
static Lock l = ...;
void raise_X()        { l.lock(); try { X_flag = true;  } finally { l.unlock(); } }
void lower_X()        { l.lock(); try { X_flag = false; } finally { l.unlock(); } }
boolean is_raised_X() { l.lock(); try { return X_flag;  } finally { l.unlock(); } }
\end{lstlisting}

\begin{tabular}{p{7cm}p{7cm}}
    \begin{lstlisting}[
    linebackgroundwidth = 18 em,
    linebackgroundcolor={%      
      \btLstHL<2>{1}
      \btLstHL<3>{2}
      \btLstHL<4>{3}
      \btLstHL<5>{6}
      \btLstHL<6>{7}      
    }]
public void useful_A() {   
  raise_A();              // A.1 
  while (is_raised_B()) { // A.2
    continue;             // A.3      
  }
  critical_section();     // A.4
  lower_A();              // A.5
 }
    \end{lstlisting}
        
          &
    \begin{lstlisting}
public void useful_B() {   
  raise_B();              // B.1
  while (is_raised_A()) { // B.2    
    lower_B();            // B.3
    while (is_raised_A());// B.4
    raise_B();            // B.5
  }
  critical_section();     // B.6
  lower_B();              // B.7
 }
\end{lstlisting} \\
\end{tabular}
\end{frame}







\begin{frame}[fragile]{Mutual exclusion solved with flags}
\framesubtitle{Thread B only}

\begin{lstlisting}
static boolean A_flag = false, B_flag = false;
static Lock l = ...;
void raise_X()        { l.lock(); try { X_flag = true;  } finally { l.unlock(); } }
void lower_X()        { l.lock(); try { X_flag = false; } finally { l.unlock(); } }
boolean is_raised_X() { l.lock(); try { return X_flag;  } finally { l.unlock(); } }
\end{lstlisting}

\begin{tabular}{p{7cm}p{7cm}}
    \begin{lstlisting}
public void useful_A() {   
  raise_A();              // A.1 
  while (is_raised_B()) { // A.2
    continue;             // A.3      
  }
  critical_section();     // A.4
  lower_A();              // A.5
 }
    \end{lstlisting}
        
          &
    \begin{lstlisting}[
    linebackgroundwidth = 18 em,
    linebackgroundcolor={%      
      \btLstHLG<2>{1}
      \btLstHLG<3>{2}
      \btLstHLG<4>{3}
      \btLstHLG<5>{8}
      \btLstHLG<6>{9}      
    }]
public void useful_B() {   
  raise_B();              // B.1
  while (is_raised_A()) { // B.2    
    lower_B();            // B.3
    while (is_raised_A());// B.4
    raise_B();            // B.5
  }
  critical_section();     // B.6
  lower_B();              // B.7
 }
\end{lstlisting} \\
\end{tabular}

\end{frame}






\begin{frame}[fragile]{Mutual exclusion solved with flags}
\framesubtitle{Contention}

\begin{lstlisting}
static boolean A_flag = false, B_flag = false;
static Lock l = ...;
void raise_X()        { l.lock(); try { X_flag = true;  } finally { l.unlock(); } }
void lower_X()        { l.lock(); try { X_flag = false; } finally { l.unlock(); } }
boolean is_raised_X() { l.lock(); try { return X_flag;  } finally { l.unlock(); } }
\end{lstlisting}

\begin{tabular}{p{7cm}p{7cm}}
    \begin{lstlisting}[
    linebackgroundwidth = 16 em,
    linebackgroundcolor={%
      \btLstHL<2-3>{1}
      \btLstHL<4-5>{2}
      \btLstHL<6>{3}
      \btLstHL<7>{4}
      \btLstHL<8>{3}
      \btLstHL<9-12>{4}
      \btLstHL<13>{3}
      \btLstHL<14>{6}
      \btLstHL<15>{7}
    }]
public void useful_A() {   
  raise_A();              // A.1 
  while (is_raised_B()) { // A.2
    continue;             // A.3      
  }
  critical_section();     // A.4
  lower_A();              // A.5
 }
    \end{lstlisting}
        
          &
    \begin{lstlisting}[
    linebackgroundwidth = 16 em,
    linebackgroundcolor={%
      \btLstHLG<3-4>{1}
      \btLstHLG<5-9>{2}
      \btLstHLG<10>{3}
      \btLstHLG<11>{4}
      \btLstHLG<12-16>{5}
      \btLstHLG<17>{6}
      \btLstHLG<18>{3}
      \btLstHLG<19>{8}
    }]
public void useful_B() {   
  raise_B();              // B.1
  while (is_raised_A()) { // B.2    
    lower_B();            // B.3
    while (is_raised_A());// B.4
    raise_B();            // B.5
  }
  critical_section();     // B.6
  lower_B();              // B.7
 }
\end{lstlisting} \\
\end{tabular}

\end{frame}


\begin{frame}[fragile]{Mutual exclusion and deadlock-freedom}

{\tiny\url{https://github.com/Svazars/parallel-programming/blob/main/hw/block1/2.2/readme.markdown}}

\begin{homeworkmail}{Task \taskProofMutex}
    Prove that algorithm on previous slide guarantees mutual exclusion for 2 threads.
    Assume it is not and get a contradiction.
}
\end{homeworkmail}

\begin{homeworkmail}{Task \taskProofDeadlockFreedom}
    Prove that algorithm on previous slide is free of deadlocks.   
}
\end{homeworkmail}

\textbf{Suggested reading}: use companion slides for ''Herlihy, Shavit: The Art of Multiprocessor Programming''\footnote{\url{https://booksite.elsevier.com/9780123973375}}, Lecture slides, Chapter 01, slides 43-72.

\end{frame}


\begin{frame}[t,fragile,noframenumbering]{Mutex basics}

\begin{minted}{java}
interface Lock { 
    void lock(); 
    void unlock(); 
}
\end{minted}

\begin{itemize}
    \item Only one contending thread enters critical section. \textbf{Mutual exclusion}.
    \item Mutex affects thread scheduling.    
    \item At least one contending thread enters critical section. \textbf{Deadlock-freedom}.
\end{itemize}

\end{frame}



\begin{frame}[fragile]{Mutual exclusion solved with flags}
\framesubtitle{Starvation}

\only<23->  {\textbf{Starvation}: user of concurrent object \textit{could} be delayed for \textit{arbitrary} time if there are other users of the same object.} \only<24->  {\textbf{Unfair mutex}: current thread starves, whole system progresses. }

\begin{tabular}{p{7cm}p{7cm}}
    \begin{lstlisting}[
    linebackgroundwidth = 16 em,
    linebackgroundcolor={%
      \btLstHL<2-3>{1}
      \btLstHL<4-5>{2}
      \btLstHL<6-9>{3}
      \btLstHL<10>{6}
      \btLstHL<11>{7}
      \btLstHL<12-13>{1}
      \btLstHL<14>{2}
      \btLstHL<15-18>{3}
      \btLstHL<19>{6}
      \btLstHL<20>{7}
    }]
public void useful_A() {   
  raise_A();              // A.1 
  while (is_raised_B()) { // A.2
    continue;             // A.3      
  }
  critical_section();     // A.4
  lower_A();              // A.5
 }
    \end{lstlisting}
        
          &
    \begin{lstlisting}[
    linebackgroundwidth = 16 em,
    linebackgroundcolor={%
      \btLstHLG<3-4>{1}
      \btLstHLG<5-6>{2}
      \btLstHLG<7>{3}
      \btLstHLG<8>{4}
      \btLstHLG<9-12>{5}
      \btLstHLG<13-15>{6}
      \btLstHLG<16>{3}
      \btLstHLG<17>{4}
      \btLstHLG<18-20>{5}
    }]
public void useful_B() {   
  raise_B();              // B.1
  while (is_raised_A()) { // B.2    
    lower_B();            // B.3
    while (is_raised_A());// B.4
    raise_B();            // B.5
  }
  critical_section();     // B.6
  lower_B();              // B.7
 }
\end{lstlisting} \\
\end{tabular}

\only<1-9>  {\texttt{Thread A acquisitions: 0, Thread B acquisitions: 0}}
\only<10-18>{\texttt{Thread A acquisitions: 1, Thread B acquisitions: 0}}
\only<19-21>{\texttt{Thread A acquisitions: 2, Thread B acquisitions: 0}}
\only<22>   {\texttt{Thread A acquisitions: N, Thread B acquisitions: 0}}

\end{frame}

\questiontime{What is the difference between \textbf{deadlock-freedom} and \textbf{starvation-freedom}?}
