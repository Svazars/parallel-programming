\documentclass[a4paper, 12pt]{extarticle}


\newcommand{\orgNum}{0}
\newcommand{\orgTopic}{org meeting}
\newcommand{\orgKey}{syllabus, contacts}

\newcommand{\introNum}{1}
\newcommand{\introTopic}{introduction to multithreading}
\newcommand{\introKey}{concurrency, parallelism, agents, threads, scheduler, Amdahl's law, race condition, deadlock, wait-for graph}

\newcommand{\basicNum}{2}
\newcommand{\basicTopic}{basic concepts}
\newcommand{\basicKey}{mutex, acquisition order, reentrancy, fairness, data locking, code locking, signalling, condition variable, lost signal, spurious wakeup}

\newcommand{\syncPrimitivesNum}{3}
\newcommand{\syncPrimitivesTopic}{advanced synchronization primitives}
\newcommand{\syncPrimitivesKey}{monitor, latch, barrier, thundering herd, semaphore, read-write lock, thread pool, executor, producer-consumer, fork-join, load balancing}

\newcommand{\patternsNum}{4}
\newcommand{\patternsTopic}{advanced synchronization concepts}
\newcommand{\patternsKey}{interruption, cancellation, partitioning, privatization, replication, thread-local, ownership}

\newcommand{\extraBasicsNum}{5}
\newcommand{\extraBasicsTopic}{additional topics of practical concurrency}
\newcommand{\extraBasicsKey}{documenting protocols and classes, checking concurrent invariants, stress testing, execution trace analysis, estimating required testing effort, static and dynamic checks, scheduling randomization, model checking}

\newcommand{\foundationsNum}{6}
\newcommand{\foundationsTopic}{theoretical foundations of concurrency}
\newcommand{\foundationsKey}{timeline, events, precedence, 2-thread mutual exclusion, deadlock freedom, starvation freedom, N-thread mutual exclusion, sequential objects and specifications, concurrent objects, linearizability}

\newcommand{\foundationsPlusNum}{7}
\newcommand{\foundationsPlusTopic}{progress guarantees, concurrent operations hierarchy, consensus number}
\newcommand{\foundationsPlusKey}{obstruction-free, lock-free, wait-free, safe register, regular register, atomic register, register snapshot, consensus number}

\newcommand{\atomicsNum}{8}
\newcommand{\atomicsTopic}{introduction to atomics}
\newcommand{\atomicsKey}{read-modify-write, get-and-add, compare-and-swap, spin lock, lock-free stack, ABA problem}

% TODO: taxonomy of queues, 

\newcommand{\cacheCoherencyNum}{9}
\newcommand{\cacheCoherencyTopic}{cache coherency}
\newcommand{\cacheCoherencyKey}{cache memory hierarchy, cache coherency protocol, store-buffer, load-buffer, invalidate-queue, memory barrier, hardware memory model, weak memory model, litmus tests}

\newcommand{\langMMNum}{10}
\newcommand{\langMMTopic}{language memory model}
\newcommand{\langMMKey}{compiler optimizations, compiler barriers, language memory model, strict consistency, threads cannot be implemented as a library, visibility, volatile}

\newcommand{\advancedConcurrencyNum}{11}
\newcommand{\advancedConcurrencyTopic}{advanced locking}
\newcommand{\advancedConcurrencyKey}{Anderson Queue Lock, CLH, MCS, check-then-act, spin-then-park}

% TODO: work distribution, work stealing, taxonomy of parallel problems, single LIFO cell optimization, RAT

\newcommand{\userSpaceThreadingNum}{12}
\newcommand{\userSpaceThreadingTopic}{user-space threading}
\newcommand{\userSpaceThreadingKey}{berkley socket, blocking and non-blocking IO, callback-hell, async-await, continuation-passing-style, fibers/coroutines/green threads, stackful vs stackless}


\newcommand{\designNum}{13}
\newcommand{\designTopic}{designing concurrent systems}
\newcommand{\designKey}{park/unpark, synchronizer, futex/wait-on-address, plan9 approach, race-finders, ForkJoinPool/CoroutineCarriers/UIthread, observability, structured concurrency}

\newcommand{\frameworksAndDistributedNum}{14}
\newcommand{\frameworksAndDistributedTopic}{multi-agent systems}
\newcommand{\frameworksAndDistributedKey}{auto-parallelization languages and frameworks, semi-automatic synchronization, distributed systems, consensus protocols}


\usepackage[utf8]{inputenc}
\usepackage[russian]{babel}

\usepackage{enumerate}

\usepackage{hyperref}
\usepackage{multirow} 
\usepackage{graphicx}
\usepackage{bm}
\usepackage{geometry}
\usepackage{listings}
\usepackage{tabularx}
\usepackage{tikz}
\usetikzlibrary{automata,positioning}

\geometry{a4paper,top=1.5cm,bottom=1.5cm,bindingoffset=0cm}
\geometry{left=2cm,textwidth=17cm}
% \linespread{1.0}
  
\usepackage{amssymb}
\usepackage{amsmath}
\usepackage{amsthm}
\usepackage{verbatim}

\title{}
\author{}
\date{}


\newcommand{\threadSafetySimple}{Неформальный подход к потоковой безопасности (thread-safety), примеры. Ключевые свойства многопоточного алгоритма: безопасность/корректность (safety), живность (liveness), производительность.}

\newcommand{\mutexBasics}{Взаимное исключение. Критическая секция (mutex). Admission policy. Реентерабельность (reentrancy), честность (fairness), голодание (starvation).}
\newcommand{\condVarBasics}{Передача сигнала от одного потока другому (signalling). Условная переменная (condition variable).}

\newcommand{\mutexDesign}{Блокировка кода (code locking), блокировка данных (data locking). Расщепление мьютексов (lock splitting). Упорядочивание критических секций для предотвращения состояния тупика (lock ordering, deadlock prevention).}

\newcommand{\commonProblems}{Проблема потерянного сигнала (lost signal), устаревания условия (predicate invalidation), внезапного пробуждения (spurious wakeup), грохочущего стада (thundering herd),
инверсии приоритета (priority inversion), голодания (starvation), конвоя (lock convoy). }

\newcommand{\foundationBasics}{Формализация многопоточного исполнения: линия времени, атомарное событие (event), трасса исполнения потока (trace model), временной интервал. Частичный порядок, отношение предшествования (precedence), его свойства.}

\newcommand{\foundationMutex}{Формальное определение взаимного исключения (mutual exclusion), отсутствия тупика (deadlock freedom), отсутствия голодания (starvation freedom). }

\newcommand{\progressBasics}{Условия прогресса (progress conditions). Dependent blocking progress conditions. Примеры. Non-blocking progress conditions. Примеры.  Dependent non-blocking progress conditions. Пример. Теорема о связи условий прогресса между собой. }

\newcommand{\regBasics}{Регистр (register): емкость (capacity), поддерживаемая конкурентность (concurrency), гарантии консистентности (consistency). Пространство регистров. Wait-free реализация объекта.}

\newcommand{\perfProbForTAS}{Проблемы масштабирования многопоточных систем, вызванные работой алгоритмов когерентности кешей. Вычислительные системы, использующие шину данных (bus-based architectures). Репликация линий кеша (cache line replication). Когерентность кешей (cache coherence). Проблема массовой инвалидации данных (invalidation storm).}





\begin{document}

\section*{Вопросы по курсу "Параллельное программирование"}

\section{Лекция~\introNum: \introTopic}

Ключевые понятия: \introKey

\begin{enumerate}[\thesection .1]
	
	\item Конкурентность (concurrency) и параллелизм (parallelism). Системы разделения времени (time-sharing), многозадачность (multitasking). Процессы и потоки. Вытесняющая многозадачность и кооперативная многозадачность. Планировщик ОС: квант планирования, смена контекста, метрики эффективности планирования, стратегии планирования, алгоритм round-robin. Задачи, легко поддающиеся параллелизации (embarrassingly parallel problems). Закон Амдала.

	\item Недетерминизм многопоточного исполнения, состояние гонки (race condition), гонка данных (data race). Word tearing. Видимость (visibility). Wait-for graph. Состояние тупика (deadlock). Инверсия приоритетов.
\end{enumerate}

\section{Лекция~\basicNum: \basicTopic}

Ключевые понятия: \basicKey

\begin{enumerate}[\thesection .1]
	
	\item \threadSafetySimple 

	\mutexBasics 

	\mutexDesign	
	
	Проблема lock convoy. Задача об обедающих философах (dining philosophers problem).

	\item \threadSafetySimple 

	\condVarBasics


	Пример API: \texttt{wait/notify/notifyAll}. Пример API: \texttt{await/signal/signalAll}.
	Проблема потерянного сигнала (lost signal), устаревания условия (predicate invalidation), внезапного пробуждения (spurious wakeup). Честность (fairness). Голодание (starvation).

\end{enumerate}


\section{Лекция~\syncPrimitivesNum: \syncPrimitivesTopic}

Ключевые понятия: \syncPrimitivesKey

\begin{enumerate}[\thesection .1]
	\item \mutexBasics \condVarBasics

	\textbf{Мониторы в языке Java} (built-in monitor). Описание API. Ключевое слово \texttt{synchronized}. Admission policy. Реентерабельность (reentrancy), честность (fairness), владение (ownership). 

	\mutexDesign

	\commonProblems	

	\item \mutexBasics 

	\condVarBasics

	\textbf{CountDownLatch}.	

	\commonProblems		

	\item \mutexBasics 

	\condVarBasics

	\textbf{CyclicBarrier}.

	\commonProblems		

	\item \mutexBasics 

	\condVarBasics

	\textbf{Semaphore}.

	\commonProblems		
	
	\item \mutexBasics 

	\condVarBasics

	\textbf{ReadWriteLock}. Предпочтение читающего или пишущего потока. Усиление (read-to-write promotion) и ослабление (write-to-read downgrade).  

	\commonProblems		

	\item \mutexBasics 

	\condVarBasics

	\textbf{Thread pools}: \texttt{ThreadFactory}, \texttt{Executor}, \texttt{Future}, \texttt{ExecutorService}, \texttt{fixedThreadPool}, \texttt{singleThreadExecutor}, \texttt{cachedThreadPool}.

	Проблемы, характерные для пулов потоков: взаимные блокировки, утечки памяти, голодание, искажение владения (ownership, thread-local).
	
	\item \mutexBasics 

	\condVarBasics

	\textbf{Шаблон ''производитель-потребитель''} (producer-consumer). Балансировка нагрузки (load balancing). Проблемы балансировки нагрузки для push-based протоколов.	
	Шаблоны разделения работы: work arbitrage, work dealing, work stealing. Fork-join модель, её преимущества и недостатки.

\end{enumerate}


\section{Лекция~\patternsNum: \patternsTopic}

Ключевые понятия: \patternsKey

\begin{enumerate}[\thesection .1]
	\item Отмена задач (cancellation, interruption).
	Прерывание задач (task cancellation) и прерывание потоков (thread cancellation).
	Односторонняя отмена задач (forced cancellation). 
	Кооперативная отмена задач с помощью cancellation token.
	Кооперативная отмена задач с помощью cancellation points.
	Прерываемые (interruptible) и непрерываемые (uninterruptible) методы.
	Политики распространения отмены задач. 
	Упорядочение асинхронных событий: отмена, сигнал условной переменной, срабатывания таймера (interruption, notification, timeout ordering).	

	\item Техники декомпозиции многопоточных программ. Конечный автомат. Перенос функций (function shipping). Выделенный поток (designated thread).
	Разбиение потоков на группы. Разбиение данных на кластеры. Приватизация данных (privatization). Репликация (replication).
	Группировка однотипных событий (batching). Владение (ownership) и привязка к потоку исполнения (thread confinement).
\end{enumerate}


\section{Лекция~\extraBasicsNum: \extraBasicsTopic}

Ключевые понятия: \extraBasicsKey

\begin{enumerate}[\thesection .1]

	\item Документирование многопоточных инвариантов. Проверка инвариантов потоковой безопасности для различных структур данных.
	Assertions. Препятствия для написания надежных многопоточных программ.	 

	\item Тестирование многопоточных программ. Юнит-тестирование, стресс-тестирование. Построение и анализ трасс исполнения.
	Проектирование тестов, исполняемых в многопоточной среде. Анализ вероятности негативного сценария, оценка требуемых вычислительных ресурсов для тестирования.  
	
	\item Инструменты анализа многопоточных программ. Статический анализ и поиск дефектов. Проверки времени исполнения.
	Мониторинг и observability API. Контроль порядка исполнения конкурентной программы (scheduling control): с помощью тестов (mocking, inheritance), с помощью языковых средств (aspects, bytecode transformation),
	с помощью операционной системы (chaos mode execution). 

\end{enumerate}


\section{Лекция~\foundationsNum: \foundationsTopic}

Ключевые понятия: \foundationsKey

\begin{enumerate}[\thesection .1]

	\item \foundationBasics

	\foundationMutex

	Алгоритмы \textbf{LockOne}, \textbf{LockTwo}, алгоритм Петерсона, доказательства их основных свойств.
	
	Теорема о нижней границе: сколько необходимо независимых ячеек памяти, чтобы добиться взаимного исключения N потоков.
	
	\item \foundationBasics

	\foundationMutex

	Алгоритм \textbf{FilterLock}, доказательства его основных свойств.

	Теорема о нижней границе: сколько необходимо независимых ячеек памяти, чтобы добиться взаимного исключения N потоков.

	\item \foundationBasics

	Последовательный объект (sequential object), последовательная спецификация (sequential specification).
	Понятие о многопоточной согласованности (concurrent consistency).
	Линеаризуемое исполнение (linearizable execution), линеаризуемый объект (linearizable object), точка линеаризации (linearization point). Примеры.

\end{enumerate}


\section{Лекция~\foundationsPlusNum: \foundationsPlusTopic}

Ключевые понятия: \foundationsPlusKey

\begin{enumerate}[\thesection .1]

	\item \progressBasics

	\regBasics

	\textbf{Теоремы о выразительной силе}:
	\begin{itemize}
		\item SRSW Safe Boolean -> MRSW Safe Boolean
		\item MRSW Safe Boolean -> MRSW Regular Boolean
		\item MRSW Regular Boolean -> MRSW Regular
	\end{itemize}	 

	\item \progressBasics

	\regBasics

	\textbf{Теоремы о выразительной силе}:
	\begin{itemize}
		\item MRSW Regular -> SRSW Atomic 
		\item SRSW Atomic -> MRSW Atomic		
	\end{itemize}	 

	\item \progressBasics

	\regBasics

	Атомарный снимок состояния N регистров (atomic snapshot). Алгоритм \textbf{CleanCollect} и его недостатки, пример ABA проблемы.
	Алгоритм \textbf{SimpleSnapshot}, обоснование корректности и его основных свойств.

\end{enumerate}


\section{Лекция~\atomicsNum: \atomicsTopic}

Ключевые понятия: \atomicsKey

\begin{enumerate}[\thesection .1]

	\item \progressBasics

	Консенсус. Теорема о невозможности достичь N-потокового консенсуса с использованием только атомарных регистров. Практические следствия. Consensus number.

	Read-modify-write objects. Примеры.

	Теорема: \texttt{compareAndSet} обладает $\infty$ consensus number. Доказательство и практические следствия.

	\item Взаимное исключения с помощью циклов и read-modify-write операций. Test-and-set-lock. Test-and-test-and-set-lock.
	Exponential backoff.

	\perfProbForTAS
	

	\item \progressBasics

	Lock-free реализация потокобезопасного LIFO контейнера. Переиспользование памяти и ABA проблема.

	\perfProbForTAS


\end{enumerate}


\newcommand{\globMemOverview}{Наивные способы синхронизации глобальной памяти. Кеш-линия (cache line). Истинное и ложное разделение данных (true sharing, false sharing). Когерентность (coherence). 
Ослабленные требования к консистентности независимых ячеек памяти. }


\section{Лекция~\cacheCoherencyNum: \cacheCoherencyTopic}

Ключевые понятия: \cacheCoherencyKey

\begin{enumerate}[\thesection .1]
	\item \globMemOverview	

	Построение эффективной подсистемы памяти: кеширование, шаблон ''репликация'', протокол обмена сообщениями. MESI протокол и примеры его работы. Преимущества и недостатки. 

	\item \globMemOverview 

	MESI протокол: состояния и их семантика, виды сообщений и их классификация (синхронные/асинхронные). Буферизация записи (store buffering), требования корректности (store forwarding).

	\item \globMemOverview 
	
	MESI протокол: состояния и их семантика, виды сообщений и их классификация (синхронные/асинхронные). Буферизация чтений (load buffering). Топология шины данных (interconnect topology).

	\item \globMemOverview 

	Модель памяти процессора (hardware memory model). Слабая модель памяти (weak memory model). Барьеры памяти (memory barriers), их классификация. Преимущества и недостатки барьерного подхода к написанию корректных конкурентных программ.	

\end{enumerate}


\newcommand{\langMM}{Модель памяти языка программирования (language memory model). Мотивация, цели, сложности при написании.}


\section{Лекция~\langMMNum: \langMMTopic}

Ключевые понятия: \langMMKey

\begin{enumerate}[\thesection .1]
	\item Оптимизации компилятора: влияние на наблюдаемое поведение однопоточной и многопоточной программы. Видимость операций с памятью (visibility).
	Барьеры для оптимизаций (compiler barriers), их таксономия. Необходимость использовать специальные языковые конструкции для ''борьбы'' с оптимизациями.
	
	\item \langMM 

	Неизменяемость (immutability). Декларативный подход к описанию параллельных вычислений.
	Строгая консистентность (strict consistency) с помощью взаимного исключения.
	Строгая консистентность (strict consistency) с помощью однопоточного исполнения.
	Реализация потоков на уровне библиотеки языка: преимущества и недостатки.
	Формальная модель памяти языка программирования. Альтернативы.
	
	\item \langMM 
	
	Видимость операций с памятью (visibility). Load-acquire/store-release подход к описанию консистентности. Ключевое слово \texttt{volatile}. Рекомендации к использованию.

\end{enumerate}

\end{document}
