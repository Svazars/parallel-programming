\documentclass[a4paper, 12pt]{extarticle}

\usepackage[utf8]{inputenc}
\usepackage[russian]{babel}

\usepackage{hyperref}
\usepackage{multirow} 
\usepackage{graphicx}
\usepackage{bm}
\usepackage{geometry}

\geometry{a4paper,top=1.5cm,bottom=1.5cm,bindingoffset=0cm}
\geometry{left=2cm,textwidth=18cm}
% \linespread{1.0}
  
\usepackage{verbatim}

\title{}
\author{}
\date{}

\begin{document}

\section*{Оформление и сдача домашних заданий}

Часть заданий подробно обсуждается в слайдах лекций, не упустите подсказки. Формулировки и уточнения по оформлению конкретных заданий можно найти по ссылке \url{https://github.com/Svazars/parallel-programming/blob/main/hw}, в файлах \texttt{blockA/lectureB.taskC/readme.markdown}.

В большинстве случаев предполагается, что вы подготовите некоторое решение (исходный текст программы) и пояснение к нему (pdf файл на русском языке). Для версионирования исходных текстов рекомендуется сделать форк репозитория и в нем завести ветку с номером задачи. Версионировать пояснения не обязательно. Для некоторых задач эффективнее обсудить решение с проверяющем устно, на практическом занятии.

Выполненные решения (или ссылки на код) отправляются на почту \texttt{filatovaur@gmail.com}. Необходимо указать тему письма в формате \texttt{<Группа>, <Фамилия>, <Задача>}. Например, правильно будет указать \texttt{11111, Филатов, 1.2}. Если в течение пары дней вы не получаете ответное письмо, то напомните о нем через другой канал связи (на лекции или семинаре, через мессенджер), письмо могло быть ошибочно определено как вредоносное. Такое периодически случается даже с университетскими аккаунтами.

\subsection*{Система оценивания}

\subsubsection*{Оценка за курс}

Курс состоит из трех блоков. За каждый блок вы получаете оценку: \texttt{2, 3, 4, 5}. Если есть хоть одна двойка, то больше тройки за семестр получить нельзя. По полученным трем оценкам вычисляется среднее арифметическое и округляется к ближайшему целому, что дает итоговую оценку за семестр. Таблица ниже проясняет описанную формулу:
\begin{center}
\begin{tabular}{|ccc|c|}
\hline
 Блок A & Блок B & Блок C & Оценка за курс \\
 \hline 
 2 & 2 & 2 & 2 \\
 2 & 2 & 3 & 2 \\
 2 & 2 & 4 & 3 \\
 2 & 2 & 5 & 3 \\
 2 & 3 & 3 & 3 \\
 2 & 3 & 4 & 3 \\
 2 & 3 & 5 & 3 \\
 2 & 4 & 4 & 3 \\
 2 & 4 & 5 & 3 \\
 2 & 5 & 5 & 3 \\
 \hline
 3 & 3 & 3 & 3 \\
 3 & 3 & 4 & 3 \\
 3 & 3 & 5 & 4 \\
 3 & 4 & 4 & 4 \\
 3 & 4 & 5 & 4 \\
 3 & 5 & 5 & 4 \\
 \hline
 4 & 4 & 4 & 4 \\
 4 & 4 & 5 & 4 \\
 4 & 5 & 5 & 5 \\
 \hline
 5 & 5 & 5 & 5 \\
  \hline \hline
\end{tabular}
\end{center}

\newpage
\subsubsection*{Оценка за блок}

Оценка за блок складывается из двух компонент:
\begin{itemize}
	\item Ответ на вопрос из билета на устном экзамене (каждый билет -- три вопроса, по одному на каждый блок)
	\item Решение практических задач в течение семестра	
\end{itemize}

При ответе на вопрос из билета вы получаете одну из следующих \texttt{надбавок}: 
\begin{itemize}
	\item "Не освоил тему" (\texttt{-3})
	\item "Плохо" (\texttt{-1})
	\item "Удовлетворительно" (\texttt{0})
	\item "Достаточно" (\texttt{+1})
\end{itemize}

Каждая из домашних задач (\url{https://github.com/Svazars/parallel-programming/blob/main/hw}) имеет стоимость в баллах. Баллы за практику превращаются в \texttt{сырую оценку} согласно таблице ниже

\begin{center}
\begin{tabular}{|c|c|}
\hline
 Баллы & Сырая оценка \\
 \hline 
 0-59    & 2 \\
 60-79   & 3 \\
 80-99   & 4 \\
 100-120 & 5 \\
 >=121   & 6 \\
 \hline \hline
\end{tabular}
\end{center}

Итоговая оценка за блок = \texttt{сырая оценка} + \texttt{надбавка}, но не больше \texttt{5}. Примеры:
\begin{center}
\begin{tabular}{|ccc|c|}
\hline
 Баллы & Сырая оценка & Надбавка & Оценка за блок \\
 \hline 
 44  & 2 & +1 & 3 \\
 100 & 5 & -3 & 2 \\
 75  & 3 &  0 & 3 \\
 90  & 4 & +1 & 5 \\
 150 & 6 & -1 & 5 \\
 110 & 5 & +1 & 5 \\
 \hline \hline
\end{tabular}
\end{center}


\end{document}
